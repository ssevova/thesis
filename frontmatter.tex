%% Title

\titlepage[\small June 2018]{}%  A dissertation submitted to Northwestern University\\ for the degree of Doctor of Philosophy}

%% Abstract
\begin{abstract}%[\smaller \thetitle\\ \vspace*{1cm} \smaller {\theauthor}]
  %\thispagestyle{empty}
  A vast portion of the Universe is predicted to consist of an as-of-yet undetected, non-luminous form of matter, known as dark matter. Evidence of its existence has been corroborated by various astrophysical observations at many cosmological scales, and its relative abundance has been determined. However, knowledge of its nature and non-gravitational interactions is entireley lacking. Nonetheless, the predicted particle nature of dark matter allows for multiple complementary methods of its detection.

This work presents a search for dark matter produced in association with a top quark pair performed using the data recorded by the CMS detector at the LHC in Geneva, Switzerland during 2016. The collision center-of-mass energy of the dataset in question is 13\:\TeV, and the integrated luminosity of the dataset corresponds to $35.9\:\textrm{fb}^{-1}$. The analysis performed considers only the dilepton decay of top quark pairs. The results are interpreted using simplified models of dark matter and are compared to corresponding results from direct detection experiments. While the work does not provide evidence of the production of dark matter in association with a top quark pair in the dilepton final state from proton-proton collisions, it sets important constraints on the properties of dark matter. 
\end{abstract}


%% Declaration
\begin{declaration}
        The following dissertation is the result of my own work conducted while based at Northwestern University and CERN. Explicit references are made to acknowledge the work of others. This dissertation has not been submitted for another qualification to Northwestern University or any other university.
  \vspace*{1cm}
  \begin{flushright}
    Stanislava Sevova
  \end{flushright}
\vspace*{\fill}
\begin{center}
\copyright\:Copyright by Stanislava Sevova 2018 \\
All Rights Reserved
\end{center}

\end{declaration}


%% Acknowledgements
\begin{acknowledgements}

A great number of people deserve recognition for directly or indirectly providing me with the support required to complete this work. Of particular prominence are my advisor at Northwestern, Kristian Hahn, and the post-doctoral researcher who has provided invaluable mentorship since day one, Kevin Sung. I would like to thank Kristian for his strong leadership and constant support of my research as a graduate student. His guidance and creativity have only amplified my own passion for particle physics throughout the years, and it has truly been a rewarding experience to work with him. To Kevin, I owe an immense amount of gratitude for his patience, understanding, and attention to detail. I have been extremely fortunate to learn from someone as knowledgeable and precise as Kevin, and I am beyond grateful for all the help he has provided over the years! I will especially and genuinely miss the office camaraderie. In addition, I would like to thank other excellent CMS colleagues, Phil Harris, Nhan Tran, and Marco Trovato who provided additional support and interesting research opportunities for me to pursue.

Throughout my graduate career, I was lucky to be surrounded by a wonderful group of friends both in Chicago and Geneva, a few of whom I feel should be mentioned by name. I would like to thank my dear friend Mary Crofton for her constant ability to take my mind off research with fun activities, and also for giving me a place to stay and feeding me while in Chicago! I would like to thank my friends Rickard Strom, Sophie Baker, and Andrew Carnes for imparting a bit of their undying sense of adventure onto me, and for the laughs they always provide. Finally, I would like to thank Leonora Vesterbacka for always being there, always knowing what to say, and always having my back! I will cherish our friendship for life. 

I would be remiss if I did not acknowledge how much Doug Schaefer has rejuvenated my quality of life in recent months, and would like to thank him not only for his immense support in recent weeks, but also for the joy he has provided me in all of our adventures together, big or small. 

To conclude, above all I am deeply thankful to my family. To my younger sister, Ralitza, I am thankful for all the ways you are able to make me smile, and I truly cherish your youthful wisdom! I look forward to giving you the same love and support you've provided me with, as you reach new heights. I dedicate this work to my parents, Penka and Lubomir, to whom I cannot express in enough words how grateful I am. You have taught me to follow my passions, persevere through the obstacles, and never be afraid of the unknown. Your support is eternal, as is my love for you. Thank you.
                        
        
%  Of the many people who deserve thanks, some are particularly prominent,
%  such as my supervisor\dots
\end{acknowledgements}


%% Preface
%\begin{preface}
%        This thesis describes my research on various aspects of the \CMS particle physics program, centred around the \CMS detector and \LHC accelerator at \CERN in Geneva.
%
%  \noindent
%  For this example, I'll just mention \ChapterRef{chap:SomeStuff}
%  and \ChapterRef{chap:MoreStuff}.
%\end{preface}

%% ToC
\tableofcontents


%% Strictly optional!
%\frontquote{%
%  Writing in English is the most ingenious torture\\
%  ever devised for sins committed in previous lives.}%
%  {James Joyce}
%% I don't want a page number on the following blank page either.
\thispagestyle{empty}
