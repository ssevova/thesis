\chapter{Background processes}
\label{chap:backgrounds}

Two classes of background processes are present in this search: reducible and irreducible. For the former category, a particle in the background process may ``fake'' the signature of a particle that is expected in the signal process. On the contrary, in the case of the latter category, the final state topology of the background process yields the same expected particles as a potential signal process. A key feature of reducible backgrounds is the ability to suppress such processes by employing the selection cuts as described in Sec.~\ref{sec:selection}. Furthermore, some of the reducible background contributions are estimated using data-driven techniques. In large part, however, the dominant backgrounds in the search are estimated from simulations.

\begin{figure}
  \begin{center}
    \subfloat[][]{\label{fig:gg_tt}
      \feynmandiagram[horizontal=b to c]{
        a [particle=\(g\)] -- [gluon] b -- [gluon] c,
        d [particle=\(g\)] -- [gluon] b,
        e [particle=\(\bar{t}\)] -- [fermion] c,
        c -- [fermion] f [particle=\(t\)],
      };
    }
    \hspace{0.5cm}
    \subfloat[][]{\label{fig:gg_tt2}
      \feynmandiagram[vertical=b to d]{
        a [particle=\(g\)] -- [gluon] b,
        c [particle=\(\bar{t}\)] -- [fermion] b -- [fermion] d -- [fermion] f [particle=\(t\)], 
        e [particle=\(g\)] -- [gluon] d,
        a -- [opacity=0.0001] e,
        c -- [opacity=0.0001] f,
      };
    }
    \hspace{0.5cm}
    \subfloat[][]{\label{fig:qq_tt}
      \feynmandiagram[horizontal=b to d]{
        a [particle=\(q\)] -- [fermion] b -- [fermion] c [particle=\(\bar{q}\)],
        b -- [gluon] d,
        f [particle=\(\bar{t}\)] -- [fermion] d -- [fermion] e [particle=\(t\)],
      };
    }
  \end{center}
  \caption{Leading order \ttbar production diagrams probed at the LHC via~\protect\subref{fig:gg_tt},~\protect\subref{fig:gg_tt2} gluon fusion, and~\protect\subref{fig:qq_tt} quark-antiquark annihilation.}
\label{fig:tt2l_feyn}
\end{figure}

\section{\ttll}

SM \ttll is the dominant background contribution and is irreducible, owing to the similarity of the final state topology with the signal processes topology. At the LHC, approximately 90\% of \ttbar events are produced via gluon fusion as shown in~\FigureRef{fig:gg_tt} and ~\FigureRef{fig:gg_tt2}, in contrast to the Tevatron at Fermilab, where quark-antiquark annihilation shown in~\FigureRef{fig:qq_tt} constituted roughly 85-90\% of the relative \ttbar production. 

The theoretical uncertainties incurred at leading order (LO) in perturbative QCD are quite large for \ttbar production. In addition to the LO simulation, the \ttbar process decaying to the dilepton final state is simulated at next-to-leading order (NLO) using the \POWHEG (v2)~\cite{powheg,powheg2} generator, with the top quark mass assumed to be \mtop=172.5\:\GeV. These events are then interfaced to \Pythia (v8.2)~\cite{Sjostrand:2014zea} for parton fragmentation, hadronization, and to simulate the underlying event. As pertains to all simulated samples subsequently described, once the \ttll events are showered, the detector response is simulated using the \Geant 4 program~\cite{AGOSTINELLI2003250}. Finally, the \ttll events are normalized to the theoretical cross section calculated at next-to-next-to-leading order (NNLO) in perturbative QCD, which also includes soft-gluon resummation calculations at next-to-next-to-leading-order (NNLL)~\cite{ttxsec1,ttxsec2,ttxsec3,ttxsec4,ttxsec5}. The cross-section folds in the branching fraction of \ttbar to the dilepton final state, which is 10.5\%. The cross-section value used is $\sigma_{\ttll}=87.31\:\mathrm{pb}$.

As mentioned in Sec.~\ref{subsec:mt2ll}, the \ttll background should be suppressed below the kinematic endpoint, $M_W$, in the \mttll distribution. This would only be possible in ideal measurement conditions, however as a cause of detector and energy resolution effects, the mismeasurement of the objects in \ttll background events can contribute to values of \mttll$>\:M_W$. 

\section{Drell-Yan}
%\begin{comment}
\begin{figure}
  \begin{center}
    \subfloat[][]{\label{fig:dy1}
      \feynmandiagram[horizontal=b to d]{
        a [particle=\(q\)] -- [fermion] b -- [fermion] c [particle=\(\bar{q}\)],
        f [particle=\(\ell^{+}\)] -- [fermion] d -- [fermion] e [particle=\(\ell^{-}\)], 
        b -- [boson, edge label=\(\gamma^{*}/Z\)] d,
      };
    } 
    \subfloat[][]{\label{fig:dy2}
      \feynmandiagram[horizontal=b to c] {
        a [particle=\(q\)] -- [fermion] b -- [fermion] c -- [fermion] d [particle=\(q\)],
        e [particle=\(g\)] -- [gluon] b,
        c -- [boson, edge label=\(\gamma^{*}/Z\)] g,
        h [particle=\(\ell^{+}\)] -- [fermion] g -- [fermion] f [particle=\(\ell^{-}\)], 
        d -- [opacity=0.0001] f,
      };
    }  
    \subfloat[][]{\label{fig:dy3}
      \feynmandiagram[vertical=b to c] {
        a [particle=\(q\)] -- [fermion] b -- [fermion] c -- [fermion] d [particle=\(q\)],
        e [particle=\(g\)] -- [gluon] c,
        b -- [boson, edge label=\(\gamma^{*}/Z\)] g,
        h [particle=\(\ell^{+}\)] -- [fermion] g -- [fermion] f [particle=\(\ell^{-}\)], 
        d -- [opacity=0.0001] h,
      }; 
    } \\
    \subfloat[][]{\label{fig:dy4}
      \feynmandiagram[horizontal=b to g] {
        a [particle=\(q\)] -- [fermion] b -- [fermion] c -- [fermion] d [particle=\(\bar{q}\)],
        e [particle=\(g\)] -- [gluon] c,
        b -- [boson, edge label=\(\gamma^{*}/Z\)] g,
        h [particle=\(\ell^{+}\)] -- [fermion] g -- [fermion] f [particle=\(\ell^{-}\)], 
        e -- [opacity=0.0001] h,
      };
    } 
    \hspace{0.7cm}
    \subfloat[][]{\label{fig:dy5}
      \feynmandiagram[vertical=b to c] {
        a [particle=\(q\)] -- [fermion] b -- [fermion] c -- [fermion] d [particle=\(\bar{q}\)],
        e [particle=\(g\)] -- [gluon] b,
        c -- [boson, edge label=\(\gamma^{*}/Z\)] g,
        h [particle=\(\ell^{+}\)] -- [fermion] g -- [fermion] f [particle=\(\ell^{-}\)], 
        e -- [opacity=0.0001] f,
      };
    }
  \end{center}
  \caption{The Drell-Yan lepton pair-production process mediated by a virtual photon ($\gamma^{*}$) or Z boson at~\protect\subref{fig:dy1} $\mathcal{O}$($\alpha$) and~\protect\subref{fig:dy2},\protect\subref{fig:dy3},\protect\subref{fig:dy4},\protect\subref{fig:dy5} $\mathcal{O}$($\alpha\alpha_{s}$).}
  \label{fig:dy_feyn}
\end{figure}

From the diagrams in~\FigureRef{fig:dy_feyn}, the Drell-Yan pair-production process falls under the class of reducible backgrounds, since many of the selection criteria act to suppress processes where the selected same flavor opposite sign (SFOS) leptons are produced at the same vertex, such as from the exchange of a real Z boson or a virtual photon ($\gamma^{*}$). Namely, the requirement for the mass of the selected SFOS lepton pair to be outside of the Z mass window, 75\:\GeV$< M_{Z} <$105\:\GeV, removes a large contribution of dilepton decays stemming from real Z bosons/off-shell virtual photons. Furthermore, the low dilepton mass requirement, $M_{\ell\ell}>20\:\GeV$ suppresses the contribution from low mass decays of $J/$\psi mesons to SFOS pairs. In addition, the requirement for the event to contain at least two jets, with at least one b-tagged jet acts to eliminate contributions from~\FigureRef{fig:dy1}, where the quark-antiquark annihilation to a SFOS pair proceeds at LO in $\alpha$. The DY process is simulated at NLO using the \AMCATNLO generator~(v5.2.3.3)~\cite{Alwall:2014hca}, and thus includes contributions from higher order processes as shown in~\FigureRef{fig:dy2}-\FigureRef{fig:dy5}, where at least one jet is expected from the fragmentation and hadronization of particles emmitted in initial state radiation.

Although the relative shape of the DY contribution is taken from simulation, a data-driven process is used to estimate the normalization of this background. The signal region still contains a size-able DY contribution, meaning that exceptional DY events evading the above-mentioned Z boson mass veto tend to be accompanied by a significant amount of \ptmiss. Since the instrumental detector effects which influence this final state topology are non-trivial to simulate, it is more appropriate to use calibrated samples from data to arrive at these estimates.  

\subsection{The \Rinout method}

The method is used to predict the DY normalization, $N_{DY}$, by extrapolating from the observed DY yield inside the Z mass window (within $\pm15\:\GeV$ of $M_{Z}$), $N_{in}$, according to:

\begin{equation}
  N_{DY} = N_{in}\frac{R^{0b}_{\mathrm{MC}}}{R^{1b}_{\mathrm{MC}}\cdot R^{0b}_{\mathrm{data}}},
  \label{eq:NDY}
\end{equation}

where each quantity $R$ in Eq.~\ref{eq:NDY} is defined as the ratio of DY yields \textbf{in}side to \textbf{out}side the Z mass window, 

\begin{equation}
  \Rinout = \frac{N(|M_{\ell\ell} - M_Z|<15\:\GeV)}{N(|M_{\ell\ell} - M_Z|>15\:\GeV\mbox{ and }M_{\ell\ell}>20\:\GeV)}.
  \label{eq:Rinout}
\end{equation}

Hence, the events originally rejected by the Z veto are used to estimate the residual contributions from DY $\rightarrow e^+e^-$ and $\mu^+\mu^-$ in the remaining selected sample. The yields are computed with all other selection cuts applied. Ideally, the \Rinout in a region where the number of b-tagged jets is required to be zero would be equal to the \Rinout in a region where at least one b-tagged jet is required, such that $\Rinout^{0b} = \Rinout^{1b}$. This assumption, however, is invalid since the numerator and denominator in Eq.~\ref{eq:Rinout} differ significantly when measured in DY simulation with a looser set of selection cuts, such as the removal of the \ptmiss requirement or a looser jet multiplicity requirement. A weaker assumption is then made, which is as follows:

\begin{equation}
  \frac{\left(\Rinout^{1b}\right)_{\mbox{data}}}{\left(\Rinout^{1b}\right)_{\mbox{MC}}} = 
  \frac{\left(\Rinout^{0b}\right)_{\mbox{data}}}{\left(\Rinout^{0b}\right)_{\mbox{MC}}},
  \label{eq:Rassump}
\end{equation}

so the ratio of the measured $\Rinout^{0b}$ between data and MC should be equivalent to the ratio of the measured $\Rinout^{1b}$ between data and MC. Then the estimate for the DY normalization in the signal region as defined in Eq.~\ref{eq:NDY} is expanded into,

\begin{equation}
  \left(N^{1b}_{\mbox{out}}\right)_{\mbox{data}} =
  \frac{\left(N^{1b}_{\mbox{in}}\right)_{\mbox{data}}}{\left(\Rinout^{1b}\right)_{\mbox{data}}} = 
  \frac{\left(N^{1b}_{\mbox{in}}\right)_{\mbox{data}}}{\left(\Rinout^{1b}\right)_{\mbox{MC}}} \cdot 
  \frac{\left(\Rinout^{0b}\right)_{\mbox{MC}}}{\left(\Rinout^{0b}\right)_{\mbox{data}}}
  \label{eq:NDY_full}
\end{equation}

Thus, every quantity on the right-hand side of Eq.~\ref{eq:NDY_full} is measured. However, it should be noted that non-DY contributions are present in the measurements made in the data, and hence must be subtracted off from events that fall both inside and outside the Z mass window in the zero b-tag and the one-or-more b-tag regions (i.e. all the quantities $N^{0b}_\text{in}$, $N^{0b}_\text{out}$, $N^{1b}_\text{in}$, and $N^{1b}_\text{out}$). The non-DY contributions in the $\{0b,1b\} \otimes \{\text{in},\text{out}\}$ regions, such as \ttll, are estimated from data using opposite flavor ($e^{\pm},\mu^{\mp}$) events, that are denoted by $N^{e\mu}_\text{in}$ and $N^{e\mu}_\text{out}$. Thus, the number of events in data in each of the aforementioned regions, after the subtraction of non-DY backgrounds is,

\begin{equation}
  N = N^{\ell\ell} - 0.5\cdot k_{\ell\ell} \cdot N^{e\mu},
\end{equation}
where the 0.5 factor accounts for combinatorics, and $k_{\ell\ell}$ is a correction factor applied to account for the differences in reconstruction efficiencies between electrons and muons. The correction factor is derived from an inclusive selection targeting $Z\rightarrow\ell\ell$, and is defined as,

\begin{equation}
  k_{ee} = \sqrt{\frac{N^{ee}}{N^{\mu\mu}}}, \hspace{0.2cm} k_{\mu\mu} = \sqrt{\frac{N^{\mu\mu}}{N^{ee}}}
\end{equation}

The value for $k_{ee} (k_{\mu\mu})$ measured in data is 0.64 (1.55). 

In order to capture any \ptmiss dependence of the DY normalization, the various \Rinout quantities are computed in four bins of \ptmiss, since the relative contribution of DY is expected to drop off at higher \ptmiss values and incur larger statistical uncertainties in the simulation.


\begin{table}[!htbp]
  \caption{DY yields and \Rinout values in the $ee$ channel, for 0 b-tag selection}
  \scalebox{0.85}{
    \begin{tabular}{l|l|c|c|c}
      \hline
      \multicolumn{2}{c|}{}                & $|M_{\ell\ell} - M_Z| < 15\:\GeV$ & $|M_{\ell\ell} - M_Z| > 15\:\GeV$ & $\Rinout^{0b}$ \\ \hline
\multirow{2}{*}{$50\:\GeV<\ptmiss<75\:\GeV$} & data & 35602.72 $\pm$ 191.00  & 4912.88 $\pm$ 92.65 & 7.25 $\pm$ 0.14\\
                                             & MC   & 38417.99 $\pm$ 233.36  & 4932.28 $\pm$ 155.12& 7.79 $\pm$ 0.25 \\ \hline
\multirow{2}{*}{$75\:\GeV<\ptmiss<100\:\GeV$} & data & 4503.12 $\pm$ 72.21  & 875.04 $\pm$ 61.05 & 5.15 $\pm$ 0.37   \\
                                             & MC    & 5651.58 $\pm$ 86.47  & 865.83 $\pm$ 58.83 & 6.53 $\pm$ 0.45 \\ \hline
\multirow{2}{*}{$100\:\GeV<\ptmiss<150\:\GeV$} & data & 714.20 $\pm$ 37.79  & 415.24 $\pm$ 56.38 & 1.72 $\pm$ 0.25  \\
                                             & MC     & 746.41 $\pm$ 31.32  & 225.78 $\pm$ 21.53 & 3.31 $\pm$ 0.34 \\ \hline
\multirow{2}{*}{$150\:\GeV<\ptmiss<1000\:\GeV$} & data & 221.68 $\pm$ 22.05 & 415.24 $\pm$ 56.38 & 0.53 $\pm$ 0.090 \\
                                             & MC      & 55.27 $\pm$ 7.33  & 105.28 $\pm$ 11.92  & 0.24 $\pm$ 0.040\\ \hline
    \end{tabular}
  }
\end{table}


\begin{table}[!htbp]
  \caption{DY yields and \Rinout values in the $\mu\mu$ channel, for 0 b-tag selection}
  \scalebox{0.85}{
  \begin{tabular}{l|l|c|c|c}
    \hline
        \multicolumn{2}{c|}{}                & $|M_{\ell\ell} - M_Z| < 15\:\GeV$ & $|M_{\ell\ell} - M_Z| > 15\:\GeV$ & $\Rinout^{0b}$ \\ \hline
\multirow{2}{*}{$50\:\GeV<\ptmiss<75\:\GeV$} & data    & 76878.78 $\pm$ 282.38 & 11061.48 $\pm$ 151.71 & 6.95 +/- 0.099 \\
                                             & MC      & 84516.00 $\pm$ 353.40 & 12266.77 $\pm$ 277.25 & 6.89 +/- 0.16\\ \hline
\multirow{2}{*}{$75\:\GeV<\ptmiss<100\:\GeV$} & data   & 9757.90 $\pm$ 109.88 & 1551.43 $\pm$ 104.12   & 6.29 +/- 0.43 \\ 
                                             & MC      & 11972.59 $\pm$ 130.57 & 2267.89 $\pm$ 104.23  & 5.28 +/- 0.25\\ \hline
\multirow{2}{*}{$100\:\GeV<\ptmiss<150\:\GeV$} & data  & 1468.25 $\pm$ 61.59 & 401.18 $\pm$ 96.96      & 3.66 +/- 0.90\\ 
                                             & MC      & 1639.18 $\pm$ 45.61 & 646.05 $\pm$ 43.72      & 2.54 +/- 0.19 \\ \hline
\multirow{2}{*}{$150\:\GeV<\ptmiss<1000\:\GeV$} & data & 305.85 $\pm$ 34.16 & 396.34 $\pm$ 97.66       & 0.77 +/- 0.20\\
                                             & MC      & 86.42 $\pm$ 10.45 & 290.42 $\pm$ 21.26        & 0.33 +/- 0.018\\ \hline
  \end{tabular}
}
\end{table}

\begin{table}[!htbp]
  \caption{DY yields and \Rinout values in the $ee$ channel, for $\geq$1 b-tag selection}
  \scalebox{0.85}{
    \begin{tabular}{l|l|c|c|c}
      \hline
        \multicolumn{2}{c|}{}                & $|M_{\ell\ell} - M_Z| < 15\:\GeV$ & $|M_{\ell\ell} - M_Z| > 15\:\GeV$ & $\Rinout^{1b}$ \\ \hline
\multirow{2}{*}{$50\:\GeV<\ptmiss<75\:\GeV$} & data    & 5236.16 $\pm$ 90.60 & $-$ & $-$\\  
                                             & MC      & 5132.28 $\pm$ 84.32 & 623.60 $\pm$ 58.67 &  8.23 +/- 0.79  \\ \hline
\multirow{2}{*}{$75\:\GeV<\ptmiss<100\:\GeV$} & data   & 1038.20 $\pm$ 58.76 & $-$ & $-$\\
                                             & MC      & 915.35 $\pm$ 34.19 & 137.98 $\pm$ 22.97 &6.63 +/- 1.13 \\ \hline
\multirow{2}{*}{$100\:\GeV<\ptmiss<150\:\GeV$} & data  & 289.88 $\pm$ 51.08 & $-$  & $-$\\
                                             & MC      & 193.95 $\pm$ 14.94 & 27.61 $\pm$ 8.35 & 7.02 +/- 2.19 \\ \hline
\multirow{2}{*}{$150\:\GeV<\ptmiss<1000\:\GeV$} & data & 154.72 $\pm$ 29.57 & $-$  & $-$\\
                                             & MC      & 22.96 $\pm$  5.00 & 17.32 $\pm$ 4.47 & 1.33 +/- 0.45 \\ \hline
    \end{tabular}
  }
\end{table}

\begin{table}[!htbp]
  \caption{DY yields and \Rinout values in the $\mu\mu$ channel, for $\geq$1 b-tag selection}
  \scalebox{0.85}{
    \begin{tabular}{l|l|c|c|c}
      \hline
      \multicolumn{2}{c|}{}                & $|M_{\ell\ell} - M_Z| < 15\:\GeV$ & $|M_{\ell\ell} - M_Z| > 15\:\GeV$ & $\Rinout^{1b}$\\ \hline
\multirow{2}{*}{$50\:\GeV<\ptmiss<75\:\GeV$} & data    & 10398.33 $\pm$ 141.70 & $-$ & $-$\\ 
                                             & MC      & 11001.22 $\pm$ 126.39 & 1444.20 $\pm$ 92.95 & 7.62 +/- 0.50 \\ \hline
\multirow{2}{*}{$75\:\GeV<\ptmiss<100\:\GeV$} & data   & 1689.88 $\pm$  97.73 & $-$ & $-$ \\
                                             & MC      & 1867.68 $\pm$  50.40 &  293.68 $\pm$ 38.12 & 6.36 +/- 0.84 \\ \hline
\multirow{2}{*}{$100\:\GeV<\ptmiss<150\:\GeV$} & data  & 372.47 $\pm$  89.03 & $-$ & $-$\\
                                             & MC      & 342.57 $\pm$  21.09 & 113.32 $\pm$ 16.96 & 3.02 +/- 0.49 \\ \hline
\multirow{2}{*}{$150\:\GeV<\ptmiss<1000\:\GeV$} & data & 100.40 $\pm$  49.44 & $-$ & $-$\\
                                             & MC      & 30.05 $\pm$   6.52 & 41.85 $\pm$ 9.82 & 0.72 +/- 0.23\\ \hline
    \end{tabular}
  }
\end{table}


\begin{table}[!htbp]
  \caption{Signal region DY yields in simulation and from \Rinout prediction in data in the $ee$ channel}
  \begin{tabular}{l|c|c|c}
    \hline
                                     & $(N^{1b}_\text{out})_\text{MC}$ & $(N^{1b}_\text{out})_\text{data}$ & scale factor \\ \hline
    $50\:\GeV<\ptmiss<75\:\GeV$      & 623.60 $\pm$  58.67         & 683.83 $\pm$ 13.85         & 1.10 $\pm$ 0.11 \\ 
    $75\:\GeV<\ptmiss<100\:\GeV$     & 137.98 $\pm$  22.97         & 198.51 $\pm$ 13.65           & 1.44 $\pm$ 0.26 \\
    $100\:\GeV<\ptmiss<150\:\GeV$    & 27.61 $\pm$   8.35          & 79.32 $\pm$ 17.34          & 2.87 $\pm$ 1.07 \\
    $150\:\GeV<\ptmiss<1000\:\GeV$   & 17.32 $\pm$   4.47          & 53.58 $\pm$ 13.66         & 3.09 $\pm$ 1.12 \\\hline
  \end{tabular}
\end{table}


\section{\ttV, diboson processes, and single top}

\section{Fake lepton background}
