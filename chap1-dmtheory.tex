\chapter{Dark matter: Beyond the Standard Model}
\label{chap:DM}

%% Restart the numbering to make sure that this is definitely page #1!
\pagenumbering{arabic}

%% Note that the citations in this chapter use the journal and
%% arXiv keys: I used the SLAC-SPIRES online BibTeX retriever
%% to build my bibliography. There are also quite a few non-standard
%% macros, which come from my personal collection. You can have them
%% if you want, or I might get round to properly releasing them at
%% some point myself.

The Standard Model (SM) of particle physics, albeit a successful theory encoding the properties of elementary particles and their interactions, nonetheless has some shortcomings. For one, cosmological and astrophysical observations supply compelling evidence~\cite{Bertone:2004pz, Feng:2010gw, Porter:2011nv} for the existence of dark matter (DM), a piece of the astro-particle physics puzzle which does not fit together with the SM. In Sec.~\ref{subsec:DMintro}, evidence and motivations forthe hunt for DM are briefly detailed, while in Sec.~\ref{subsec:DMsearches} the main modes of DM detection are outlined, with particular emphasis on collider detection. In Sec.~\ref{subsec:SM}, the connection between the SM and DM is presented with particular emphasis on beyond the Standard Model (BSM) simplified DM models currently being probed at general-purpose detectors at the Large Hadron Collider (LHC) in Geneva, Switzerland.

\section{Introduction to dark matter}
\label{sec:DMintro}

Observations at all scales, from smaller dwarf galaxies to large cosmological scales point to the existence of more matter than is reconcilable with the amount of visible matter in our universe. This was first postulated by Swiss physicist Fritz Zwicky in 1933 whilst observing the Coma cluster. Zwicky's observations pointed to the need for approximately 10 times the mass as observed from the visible light of the cluster to ensure the gravitational bounding of individual galaxies within the cluster itself. Subsequently Vera Rubin's research in the 1970's revealed the flat dependence of \textit{v}, the galactic rotation velocity, as a function of the radius \textit{r} beyond the visible galactic disk demonstrating that mass extends past the visible disk. By the 1980's the majority of the astrophysical community was convinced that a substantial amount of invisible matter existed in the universe. 

Studies of the large scale structure of the universe have provided clues as to the nature of dark matter. Just as on the small scale, ordinary visible matter consists of protons, electrons, neutrons, or groups of atoms held together by the electromagnetic force, similarly groups of stars are bound together by the gravitational force provided sufficiently massive in order to form galaxies, and galaxies form clusters, and so on. The observation that star ages within galaxies are on the order of 10 to 14 billion years old, and cluster formation is still under way serves to support the cold dark matter (CDM) hypothesis. In this case, DM comprises of rather massive, slow moving, and non-relativistic particles, which would stimulate the clumping of matter into small regions initially, eventually giving rise to larger scale structures. This bottom-up theory of structure formation is further supported by myriad computer simulations consisting of billions of dark matter particles confirming the CDM model yields large structures such as those observed by the Sloan Digital Sky Survey. 

\section{Dark matter detection}
\label{sec:DMsearches}

\subsection{Direct detection}

\subsection{Indirect detection}

\subsection{Collider searches}

\section{Simplified models of DM: beyond the Standard Model}
\label{sec:SM}

%~\cite{Phys.Rev.Lett.19.1264, Phys.Rev.D2.1285,hep-ph/0410370}.

